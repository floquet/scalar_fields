\begin{myTheorem}[\label{thm:zero mean}Mean value for Zernike expansions]
Given the Zernike polynomial expansion over $\dtwo$ in \eqref{eq:Zernike expansion} the mean value of the function $\prt$ is given by
  % % % EQUATION
  \begin{equation}
    \inner{\prt}_{\dtwo} = \alpha_{0,0}.
    \label{eq:assertion}
  \end{equation}
  % % %
\end{myTheorem}
  %  x  x  x  x  x  x  x  x  x  x  x  x  x  x  x  x  x  x
\begin{proof}
Define the mean value of the function as
  % % % EQUATION
  \begin{equation}
    \inner{\prt}_{\dtwo} = \frac{\int_{\dtwo}\prt rdrd\theta} {\int_{\dtwo}rdrd\theta}.
  \end{equation}
  % - -  e q u a t i o n
Since the radial polynomials and exponential terms are continuous functions on a compact interval we may safely interchange the order of summation and integration:
  \begin{equation}
    \int_{\dtwo}\prt rdrd\theta = \sum_{n=0}^{d} \sum_{m=\odot}^{n(2)} \alpha_{n,m} \int_{0}^{1} R_{n}^{m}(r) \int_{0}^{2\pi} e^{i m \theta} d\theta r dr
  \end{equation}
  % - -
where we allow the limit $d \to \infty$.

This proof resolves two different regimes for the angular velocity $m\in\mathbb{N}$: the first $m>0$, the second $m=0$. This distinction arises from the integration of the angular variable:
  % % % EQUATION
  \begin{equation}
    \int_{0}^{2\pi} e^{im\theta} d\theta = 
    \begin{cases}
      0 & m \ne 0 \\
      2\pi & m = 0
    \end{cases} .
  \end{equation}
  % % %
The expansion terms corresponding to values of $m>0$ do not contribute to mean value. We now address the rotationally invariant terms where $m=0$ and here too we find a distinction this time for $k=0$ and $k\ge1$. The assertion in \eqref{eq:assertion} is now restated for $k\in\mathbb{N}$ as
  % - -  e q u a t i o n
  \begin{equation}
    \int_{0}^{1} R_{2k}^{0}(r) r dr = 
      \begin{cases}
        \half & k = 0 \\
        0 & k\ge 1
      \end{cases}
  \end{equation}
  % - -
The case of $k=0$ is trivial and we must exhibit the veracity of the claim for $k\ge 1$.
Integrate the terms in \eqref{eq:recursion}.
  % % % EQUATION
  \begin{equation}
    \fraction \int_{0}^{1} r^{2(k-j)} r dr = 
    \frac{1} {2\paren{k-j+1}} \fraction
  \end{equation}
  % % %
The integral for each polynomial of order $2k$ is reduced to the sum
  % % % EQUATION
  \begin{equation}
    \mathcal{S}_{k} = \int_{0}^{1} R_{2k}^{0}(r) r dr = \sum_{j=0}^{k} \frac{(-1)^{j}} {2(k-j+1)} \fraction
   \label{eq:sk}
  \end{equation}
  % % %
and we employ induction to prove $\mathcal{S}_{k} = 0$ for $k\ge1$. The base case is  $\mathcal{S}_{1} = \half - \half = 0$. For $k>1$ we assume $\mathcal{S}_{k} = 0$ and exploit this to establish $\mathcal{S}_{k+1} = 0$. The proof strategy is to demonstrate that $\mathcal{S}_{k} + \mathcal{S}_{k+1} = 0$.

The sum of consecutive sequences is
  % - -  e q u a t i o n
  \begin{equation}
    \mathcal{S}_{k} + \mathcal{S}_{k+1} = 
    \new +
    \sum_{j=0}^{k} \sign \paren{1-\frac{2(k-j)} {j+1}} \fraction 
    = 0
  \label{eq:consecutive}
  \end{equation}
  % - -
which implies, after simplification of the summation terms, that
  % - -  e q u a t i o n
  \begin{equation}
    \new =
    -\sum_{j=0}^{k} \sign \frac{\paren{k-j} \paren{k-j+1} \paren{2k-j}!} {\paren{j+1}!\facts} .
  \label{eq:zerosum}
  \end{equation}
  % - -
The sum in \eqref{eq:zerosum} can be evaluated:
  % - -  e q u a t i o n
  \begin{equation}
     \sum_{j=0}^{k} \sign \frac{\paren{k-j} \paren{k-j+1} \paren{2k-j}!} {\paren{j+1}!\facts}  = -\miracle.
  \end{equation}
  % - -
With modest manipulation we connect to \eqref{eq:consecutive}:
  % - -  e q u a t i o n
  \begin{equation}
    \miracle =
    \new.
  \end{equation}
  % - -
This substantiates $\mathcal{S}_{k} + \mathcal{S}_{k+1} = 0$, therefore $\mathcal{S}_{k+1} = 0$. Therefore \eqref{eq:sk} is valid, therefore
  % - -  e q u a t i o n
  \begin{equation}
    \int_{0}^{1} R_{2k}^{0}(r)rdr = 0, \qquad k \ge 1.
  \end{equation}
  % - -
This establishes
  % - -  e q u a t i o n
  \begin{equation}
    \sum_{k=0}^{\infty} \paren{\alpha_{2k,0} \int_{0}^{1} R_{2k}^{0}(r)rdr} = \half \alpha_{00}
  \end{equation}
  % - -
which confirms the contention in \eqref{eq:assertion}.
\end{proof}

\endinput %  .  .  .  .  .  .  .  .  .  .  .  .  .  .  .  .  .