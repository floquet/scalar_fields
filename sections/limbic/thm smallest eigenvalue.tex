\begin{theorem}[Smallest eigenvalue]
In the limit $n\to\infty$, the smallest eigenvalue approaches 0 quadratically is
  \begin{equation*}   %  =   =   =   =   =
   %\begin{split}
      \lim_{n\to\infty} \lambda_{n} = n^{-2}.
   %\end{split}
   %\label{eq:}
  \end{equation*}
\label{thm:smallest eigenvalue}
\end{theorem}
%  %  %  %  %  %  %  %  %  %  %  %  %  %  %  %  %  %  %
\begin{proof}
As the $\B{}$ matrix is lower triangular, the eigenvalues are on the diagonal. By the diagonal dominance theorem in \ref{eq:limb dom}, the value of the smallest eigenvalue corresponds to $b_{nn}$. Refer to \eqref{eq:area I} using $\paren{r_{1}, r_{2}} = \paren{1, 1 - \Delta}$, $\paren{y_{1}, y_{2}} = \paren{\Delta,0}$ and apply the limit.
\end{proof}

\endinput %  .  .  .  .  .  .  .  .  .  .  .  .  .  .  .  .  .