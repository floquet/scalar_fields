\begin{myTheorem}[Dominance of the sector areas in the outer shell]  
Given a unit disk partitioned into $n\ge2$ shells and limbs, the sectors of the outer shell are ordered according to
  % = =  e q u a t i o n
  \begin{equation}
    A_{11} > A_{21} > \cdots > A_{n1}.
  \end{equation}
  % = =
\label{eq:shell dom}
\end{myTheorem}  %  -  -  -  -  -  -  -  -  -  -  -  -  -  -  -  -
\begin{proof}  %  +  +  +  +  +  +  +  +  +  +  +  +  +  +  +  +
The proof is broken into two parts. The first is to establish the special case $A_{11}>A_{21}$ then to establish the $A_{k,1}>A_{k+1,1}$ for $2\le k \le n-1$.

\textbf{Part I} $A_{11} > A_{21}$:
Call upon equations \eqref{eq:area delta} and \eqref{eq:area I} (with $y_{1} = \omtd$ and $y_{2} = \omd$; $r_{1} = \omd$ and $r_{2} = 1$):
%  % = =  e q u a t i o n
  \begin{multline}
    \paren{\omd}^{2} \paren{\arcsin\paren{\frac{\omtd}{\omd}} -  \frac{\pi}{2}} + \arccos\paren{\omd} - \arcsin \paren{\omd} \\ +\arcsin \paren{\omtd} >
    \paren{\omtd} \paren{\sssqrt - 2 \ssssqrt}
  \end{multline}
  % = =
The right-hand side is negative for $0<\Delta<\half$ so it will suffice to show that the left hand side, $f(\Delta)$, is positive over the same domain. A Maclauren expansion produces a lead term of 
  % = =  e q u a t i o n
  \begin{equation}
    f(\Delta) = \paren{\frac{3}{\sqrt{2}} - 1} \sqrt{\Delta} + \mathcal{O}(\Delta^{3/2}) .
  \end{equation}
  % = =
Near the origin this function behaves as the square root, increasing from 0 at the origin. To resolve the behavior of the function away from the origin examine the derivative
  % = =  e q u a t i o n
  \begin{equation}
    f'(\Delta) = \frac{2}{\ssqrt} + \frac{\omd}{\sssqrt} - \frac{1}{\ssssqrt} - 2 (\omd ) \arccos \paren{\frac{\omtd}{\omd}}
  \end{equation}
  % = =
which is positive until $\Delta \approx 0.41$ at which point it becomes negative. The function value at the boundary point is
  % = =  e q u a t i o n
  \begin{equation}
    f \paren{\half} =\frac{7\pi}{24} > 0.
  \end{equation}
  % = =
Thus $f(\Delta)> 0$ on the domain $\Delta \in (0,1/2]$.

\textbf{Part II} $A_{k,1} > A_{k+1,1}$: Exploit the monotonicity of the sector width using a common partition for the Riemann integral.

Let $P=\lst{x_{0}, x_{1}, \dots, x_{\mu}}$ be a partition of the interval $\brac{0,\Lambda}$ such that
%%%
\begin{equation}
  0 = x_{0} < x_{1} < \cdots < x_{\mu-1} < x_{\mu} = \Lambda
\end{equation}
%%%
with $\mu\ge2$. Define the sector width as
%%%
\begin{equation}
  w(y) = \somys - \somdmys, \qquad 0 \le y \le \oml .
\end{equation}
%%%
Inspection of the derivative
%%%
\begin{equation}
  w'(y) = \frac{y}{\somdmys} - \frac{y}{\somys}
\end{equation}
%%%
establishes that $w(y)$ increases monotonically on the domain $0 \le y \le \oml$.

Define the subdomain of the sectors (why?) as
%%%
\begin{equation}
  y_{\Lambda} = \lst{y\ir:(n-\Lambda)\Delta \le y \le (n-\Lambda+1)\Delta} .
\end{equation}
%%%
Let $t_{k}$ be a point in the subinterval $\brac{x_{k-1},x_{k}}$. The Riemann sum over the subdomain $\omega_{\Lambda,1}$ is
%%%
\begin{equation}
  S_{\Lambda} (P,w) = \smeotm w\paren{ (n-\Lambda)\Delta + t_{k} } \intrvl
\end{equation}
%%%
We will show $S_{\Lambda} (P,w) > S_{\Lambda+1} (P,w)$.

Define the numbers
\begin{equation}
  \begin{split}
    M_{\Lambda,k}(P,w) 
      &= \sup \lst{w\paren{ (n-\Lambda)\Delta+x }: x \in \subint }, \\
    m_{\Lambda,k}(P,w) &= \inf \lst{w\paren{ (n-\Lambda)\Delta+x }: x \in \subint }. 
  \end{split}
\end{equation}
Since $w(y)$ increases monotonically on $0\le y < \omd$ these numbers attain extremal values at the subdomain boundaries:
%%%
\begin{equation}
  \begin{split}
    M_{\Lambda,k}(P,w) 
      &= w\paren{ (n-\Lambda)\Delta+x_{k} }, \\
    m_{\Lambda,k}(P,w) &= w\paren{ (n-\Lambda)\Delta+x_{k-1} }. 
  \end{split}
\end{equation}
%%%
The upper and lower Riemann sums for the partition $P$ are
\begin{equation}
  \begin{split}
    U_{\Lambda}(P,w) &= \smeotm M_{\Lambda,k}(P,w) \intrvl, \\
    L_{\Lambda}(P,w) &= \smeotm m_{\Lambda,k}(P,w) \intrvl.
  \end{split}
\end{equation}
Since $w(y)$ increases monotonically we can state that for any $t_{k}\in\subint$
%%%
\begin{equation}
  m_{\Lambda,k}(P,w) \le w\paren{ (n-\Lambda)\Delta + t_{k} }\le M_{\Lambda,k}(P,w)
\end{equation}
%%%
which implies
%%%
\begin{equation}
  L_{\Lambda}(P,w) \le S_{\Lambda}(P,w) \le U_{\Lambda}(P,w).
\end{equation}
%%%
Because the arguments are ordered
%%%
\begin{equation}
  (n-\Lambda)\Delta + x_{k-1} > (n-\Lambda-1)\Delta + x_{k}
\end{equation}
%%%
the functions are ordered
%%%
\begin{equation}
  w\paren{ (n-\Lambda)\Delta + x_{k-1} } > w\paren{ (n-\Lambda-1)\Delta + x_{k} }
\end{equation}
%%%
which implies
%%%
\begin{equation}
  m_{\Lambda}(\omega) > M_{\Lambda+1}(\omega)
\end{equation}
%%%
which implies
%%%
\begin{equation}
  L_{\Lambda}(P,w) > U_{\Lambda+1}(P,w).
\end{equation}
%%%
Call $\part$ the set of all partitions with at least two subintervals. Then we see that for any $P\in\part$
%%%
\begin{equation}
  S_{\Lambda}(P,w) > S_{\Lambda+1}(P,w)
\end{equation}
%%%
which confirms
%%%
\begin{equation}
  A_{\Lambda,1} > A_{\Lambda+1,1}, \qquad \Lambda = 2, \dots, n-1 .
\end{equation}
%%%
\end{proof}

\endinput %-------------------------------