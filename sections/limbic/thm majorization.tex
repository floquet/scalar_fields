\begin{myTheorem}[Majorization of the sequence of response functions]  
Given the function sequence in \eqref{eq:family Y}, the functions are majorized according to
  % % % EQUATION
  \begin{equation}
    \mathcal{Y}_{n+2}(b) \ge \mathcal{Y}_{n}(b), \qquad n = 0, 1, 2, \dots
  \end{equation}
  % % %
  \label{th:response}
\end{myTheorem}  %  -  -  -  -  -  -  -  -  -  -  -  -  -  -  -  -
\begin{proof}  %  +  +  +  +  +  +  +  +  +  +  +  +  +  +  +  +
Define $\rho = \xpbs$. Given $b\in[0,1]$, we have $0 \le \rho \le \cost \le 1$ accordingly $\rho^{2} \le 1$. Therefore
  % % % EQUATION
  \begin{equation}
    \int_{0}^{\cost} \rho^{2} \rho \, dx_{1} \le \int_{0}^{\cost} \rho \, dx_{1} ,
  \end{equation}
  % % %
with the equality is achieved in the limit $b\to 1$.
\end{proof}  %  +  +  +  +  +  +  +  +  +  +  +  +  +  +  +  +

\endinput %-------------------------------